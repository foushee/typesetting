\documentclass[11pt]{memoir}
\usepackage[top=1in, bottom=1in, left=1.5in]{geometry}
\usepackage[utf8]{inputenc}
\usepackage[T1]{fontenc} % Required for accented characters
\usepackage{fontspec}
%\usepackage[light,math]{Calluna-Regular}
\setmainfont{[Calluna-Regular.ttf]}
\usepackage{xcolor}
\usepackage{color}
\usepackage{ulem}
\parindent0pt
\renewcommand{\baselinestretch}{1.11}
\title{grief}
\author{foushee }
\date{December 2019}

\begin{document}
\thispagestyle{empty}
\noindent \textbf{\textsc{\Large In the Year 2020}}
\vspace{18pt}

Which of us will be left then ---\\
old, dazed, unclear ---\\
but willing to talk about our dead friends?\\
Talk and talk, like an old faucet leaking.\\
So that the young ones,\\
respectful, touchingly curious,\\
will find themselves stirred\\
by the recollections.\\
By the very mention of this name\\
or that name, and what we did together.\\
(As we were respectful, but curious\\
and excited, to hear someone tell\\
about the illustrious dead ahead of us.)\\
Of which of us will they say\\ 
to their friends,\\
he knew so and so! He was friends with \rule{1cm}{0.15mm}\\
and they spent time together.\\
He was at that big party.\\
Everyone was there. They celebrated\\
and danced until dawn. They put their arms\\
around each other and danced\\
until the sun came up.\\
Now they're all gone.\\
Of which of us will it be said ---\\
he knew them? Shook hands with them\\
and embraced them, stayed overnight\\
in their warm houses. Loved them!\\

Friends, I do love you, it's true.\\
And I hope I'm lucky enough, privileged enough,\\
to live on and bear witness.\\
Believe me, I'll say only the most\\
glorious things about you and our time here!\\
For the survivor there has to be something\\
to look forward to. Growing old,\\
losing everything and everybody.

\vspace{22pt}
\hspace{62pt} \textsc{Raymond Carver}\\
\vfill
\noindent\footnotesize{

\textcolor{gray}{Copyright © 1993 by Raymond Carver.}} 
\end{document}
