\documentclass{memoir}
\usepackage[top=.8in, bottom=.8in, left=1.5in]{geometry}
\usepackage[utf8]{inputenc}
\usepackage[T1]{fontenc} % Required for accented characters
\usepackage{fontspec}
\setmainfont{[Calluna-Regular.ttf]}
\usepackage{xcolor}
\usepackage{color}
\renewcommand{\baselinestretch}{1.05}
\parindent0pt

\title{clothes_Neruda_incomplete}
\author{Ruthe Foushee}
\date{February 2020}

% INCOMPLETE
\begin{document}
\thispagestyle{empty}
\noindent \textbf{\textsc{\Large Oda al Traje}}
\vspace{18pt}

Cada ma\~{n}ana espreas\\
Traje, sobre una silla\\
Que te llene\\
Mi vanidad, mi amor\\
Mi esperanza, mi cuerpo\\
Apenas\\
Salgo del sueño\\
Me despido del agua\\
Entro en tus magas\\
Mis piernas buscan\\
El hueco de tus piernas\\
Y así embrazado\\
Por tu fidelidad infatigable\\
Salgo a pisar en el pasto\\
Entro en la poesía\\
Miro por las ventanas\\
Las cosas\\
Los hombres, las mujeres\\
Los hechos y las luchas\\
Me van formando\\
Me van haciendo frente\\
Labrandome las manos\\
Abriéndome los ojos\\
Gastándome la boca\\
Y así,\\
Traje,\\
Yo también voy formándote\\
Sacandote los codos\\
Rompiéndote los hilos\\
Y así tu vida crece\\
A imagen de mi vida\\
Al viento\\
Ondulas y resuenas\\
Como sifueras mi alma.\\
En los malos minutos\\
Te ad hieres\\
A mis huesos\\
Vacío, por la noche\\
La oscuridad, el sueño\\
Pueblan con sus fantasmas\\
Tus alas y las mías\\
Yo pregunto\\
Si un día\\
Una bala\\
Del enemigo\\
Te dejara una mancha de mi sangre\\
Y entonces\\
Morrirías conmigo\\
O tal vez\\
No sea todo\\
Tan dramático\\
Sino simple\\
Y te irás enfermando,\\
Traje,\\
Conmigo\\
Envejeciendo\\
Conmigo, con mi cuerpo\\
Y juntos\\
Entraremos\\
A la tierra.\\
Por eso\\
Cada día\\
Te saludo\\
Con reverencia y luego\\
Me abrazas y te olivido\\
Porque uno solo somos\\
Y seguiremos siendo\\
Frente al viento, en la noche,\\
Las calles, ola hucha\\
Un solo cuerpo\\
Tal vez, tal vez, una vez immóvil.\\

\vspace{11pt}
\hspace{80pt} \textsc{Pablo Neruda}\\
\end{document}
