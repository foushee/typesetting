\documentclass[14pt]{memoir}
\usepackage[top=.9in, bottom=.75in, left=1.5in]{geometry}
\usepackage[utf8]{inputenc}
\usepackage[T1]{fontenc} % Required for accented characters
\usepackage{fontspec}
%\usepackage[light,math]{Calluna-Regular}
\setmainfont{[Calluna-Regular.ttf]}
\usepackage{xcolor}
\usepackage{color}
\usepackage[absolute,overlay]{textpos}
\usepackage{setspace}
%\onehalfspacing
%\doublespacing
\renewcommand{\baselinestretch}{1.1}
% intention of a paragraph
\parindent0pt

\title{grief}
\author{foushee }
\date{December 2019}

\begin{document}
\thispagestyle{empty}
\textbf{\textsc{\large I don't want to be a spice store}}
\vspace{32pt}

I don't want to be a spice store.\\
I don't want to carry handcrafted Marseille soap,\\
or tsampa and yak butter,\\
or nine thousand varieties of wine.\\
Half the shops here don't open till noon\\
and even the bookstore's brined in charm.\\
I want to be the one store that's open all night\\
and has nothing but necessities.\\
Something to get a fire going\\
and something to put one out.\\
A place where things stay frozen\\
and a place where they are sweet.\\
I want to hold within myself the possibility\\
of plugging one's ears and easing one's eyes;\\
superglue for ruptures that are,\\
one would have thought, irreparable,\\
a whole bevy of non-toxic solutions\\
for everyday disasters. I want to wait\\
brightly lit and with the patience\\
I never had as a child\\
for my father to find me open\\
on Christmas morning in his last-ditch, lone-wolf drive\\
for gifts. ``Light of the World'' penlight,\\
bobblehead compass, fuzzy dice.\\
I want to hum just a little with my own emptiness\\
at 4 a.m. To have little bells above my door.\\
To have a door.\\

\vspace{12pt}
\hspace{90pt} \textsc{Christian Wiman}\\
\vfill
\noindent\footnotesize{
%Matthew Dickman, ``Grief'' from All-American Poem. 
\textcolor{gray}{Copyright © 2019 by Christian Wiman.}} %Reprinted by permission of American Poetry Review and the author.
\end{document}